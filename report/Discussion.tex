\section{Discussion}
In the following chapter, the results gathered will be interpreted, focusing on addressing the specified research question. This analysis will enable the hypothesis to be tested, determining whether it holds true or not.

\subsection{Data Interpretation}

The experiment aimed to investigate the relationship between the training iteration count of a machine learning model against the accuracy of a computer vision program. The results illustrated a noticeable exponential increase in the correlation between accuracy and the training iteration count. With advancements in the model's training, it was able to continuously improve its accuracy. This demonstrates the machine learning self-improvement as it approaches a state of convergence. This illustrates a strong correlation between the given independent and dependent variables. \\

When analysing the model's evolution, it is observed that during the initial training phase, the model either failed to identify any objects or did so with a limited accuracy score. This can be justified by the model's unfamiliarity with the objects it is intended to identify, as it is not yet familiar with the given patterns through which it should detect the objects. In the next stage, the model demonstrates rapid development as its accuracy scores vastly increase, displaying an improvement in the model's object detection. This shows the model establishing an understanding of the objects and further developing its pattern recognising, resulting in improved accuracy scores. As the experiment progresses into the later stages, the model begins to approach a state of convergence. Additionally, the trade-off between training and accuracy becomes less pronounced. Increased training no longer results in further increased accuracy, demonstrating the reduced effectiveness of the trade-off. Herby also indicates the beginning of a plateauing stage for the accuracy data.  \\ 

Noteworthy is also the weak inverse correlation between the loss and accuracy of the model throughout the training process. This indicates that as the accuracy increases, the loss decreases. The loss is a measure of the errors made by a model. This also explains the observed inverse correlation. With fewer errors, the lower the loss will be and in return will lead to higher precision, resulting in a higher accuracy score. \\


Additionally, the findings reveal a slight decline in the performance of the model in the later stages of the experiment, characterised by a dip in the results. An explanation for this can be overfitting, in which the model excessively adapts to the given training data, compromising the ability to adapt to new data such as the test images used. Therefore, after overfitting, the model could obtain lower accuracy scores in comparison to scores obtained in previous training iterations. \\

Notable is also the spread of the data and its variance found in the results as the accuracy improves over time. This is a key aspect of such models as for real-world applications, where consistent performance is essential. For instance, in self-driving cars, such deployed models must be able to consistently identify objects with high precision, not just occasionally. Even though the model at times reached high accuracy scores, it was inconsistent and instances of low accuracy scores were recurring, resulting in greater variance. However, in the later training stages,  as the median accuracy increased, the variance reduced, resulting in the lower accuracy scores of the given later stage, still being relatively high and precise predictions by the model.   \\

As revealed in the results section two distinct outliers were discovered in the ‘dog’ and ‘wheel’ image categories. These data points, significantly deviate from the. To mitigate the impact such outliers have, a larger pool of data points would be helpful. \\

\subsection{Hypothesis and Research Question Evaluation}
Following the interpretation of the results obtained, it is of importance, to evaluate the key postulations formed in the hypothesis. It was hypothesised that the accuracy of the model would increase as the training iteration count advanced, this correlation between the two variables of the experiment was portrayed. Further, it was proposed, that during the initial training iterations of the experiment, the data would portray little or no accuracy as the model would be unfamiliar with the patterns it is analysing. This likewise was confirmed by the data. Furthermore, it was predicted that between 3000 and 12000 the largest growth of accuracy would be discovered, which again was confirmed in the results\\


A further hypothesis stated that after 80\% accuracy, the data would start plateauing out. This postulation was disproved as the accuracy carried on improving and only in the later stages of the experiment, whilst over 95\% started plateauing out.  Additionaly the hypothesised inverse correlation between loss and accuracy was also confirmed. As the model's accuracy improved, the number of errors decreased, which led to a reduced loss.\\ 

In conclusion, it can be stated that the hypothesis set forth for the experiment was to a high degree accurate and received validation from the data obtained, offering insightful conclusions for the given research question. 	\\	

\subsection{Limitations}
The first limitation faced was regarding the constrained data set available. This was portrayed by the model only being evaluated up to 20,000 trainings iterations, leaving the model's performance thereafter unknown. This meant that there was no data on how the model functions after the 20,000 iterations. This limitation restricted the possibility of evaluating patterns that could be found thereafter.\\ 

A further limitation of the given experiment is the limited computing power a M1 Macbook Air offers. Whilst this laptop offers acceptable computing capabilities, it is restricted when it comes to extensive data sets and complex computations. This can have direct effects on the efficiency and depth of the experiment.\\

The final limitation faced in this experiment concerns itself with the machine learning model created using apples Create ML. Given the lack of control and insight into the internal functions of Create ML, the possibility of customising the model according to the requirements is limited. Therefore having the given dependency on a pre-designed tool restricts the possible flexibility and addptability of the experiment following the given methodology. \\

\subsection{Further research}
Following the experiment concerning the relationship between accuracy and training iteration count, further research possibilities emerged. Here two such opportunities will be outlined with a short rationale. \\

The first further research opportunity concerns itself with the kernel stride length in the computer vision programs' pattern recognition. An investigation could be conducted, experimenting with how different stride lengths impact the accuracy of such a model.  \\

The second area for further research concentrates on speed vs accuracy and the exact, trade-off between these two. This experiment could aim to find an optimal zone in which the training duration is manageable, whilst still aching acceptable accuracy.  \\




