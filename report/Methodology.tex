\section{Methods}
This chapter covers the methodology used to conduct the expermiment which is the main focus of this paper.
It will cover the research design, the experiment setup and execution, as well as the data transformation 
and visualization.

\subsection{Research Design}
The research of this paper is based on empirical and quantitative data collected from an experiment.
As a foundation for this experiment, there are key variables one being the independent variable and the 
dependent variable. The independent variable in this case is the number of iterations the algorithm will
process the dataset, while the dependent variable is the accuracy of the algorithm in detecting objects.
To make sure results are accurate and consistent, all ierations will use the same machine learning algorithm,
the same dataset, the same evaluation data, and run on the same hardware.
The data which is being collected and analyzed is the number of iterations (continuous ratio), the accuracy
and loss (continuous ratio), and the relation between them.

\subsection{Experiment Setup and Execution}
The program "CreateML" from \cite{Apple} was used to carry out this experiment. It is a user-friendly and powerful 
tool that allows developers to create custom machine learning models without the need for extensive expertise in the field.
CreateML simplified the process of creating and training an object detection model by providing a graphical user interface which 
just takes the input dataset and the desired parameters. The object detection model uses the YOLOv2 \parencite{Jain} algorithm.
"You Only Look Once (YOLO) is a state-of-the-art, real-time object detection algorithm introduced in 2015" \parencite{Keita2022}. 
The dataset used to train the object detection model is \citetitle{pascal2023} which is a collection of around 10000 images including classes
like persons, vehicles and animals. The experiment was ran on a MacBook Air 2020 with an Apple M1 chip and 8GB of RAM. To make sure the 
result was usable the "Batch size" was left on auto and the "Grid Size" was left on 13x13.\\
\newpage
Overall the model was trained 20.000 times and every 1000 iterations a snapshot of the model was taken and evaluated using
specific images. 
The classes and images chosen include: Boat, Dog, Person, Plant, Wheel.
\begin{figure}[h]
  \centering
  \includegraphics[width=0.70\textwidth]{../Data/distribution-classes-barchart.png}
  \caption{Class Usage in Dataset (\%)}
\end{figure}
\\
Each snapshot was evaluated by using the above mentioned images and the accuracy as well as the loss was saved in a CSV file to later 
analyse and visualise.

\subsection{Data Transformation and Visualisation}
The data transformation and visualisation is done by...

